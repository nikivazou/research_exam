\section{Abstract Refinement Types}\label{sec:abstract}

Refinement type systems, as presented so far, fall into two categories.
Expressive types systems, as presented in Section \ref{sec:undec}, 
are statically undecidable; while decidable systems, as presented in Section \ref{sec:liquid}, 
restrict the refinement language in a subset on first order logic.
%
In this section, we present abstract refinement types\cite{Vazou13},
a means to enhance expressiveness of a refinement system,
while preserving (SMT-based) decidability.
%
The key insight is that we avail 
quantification over
the refinements of data- and function-types,
simply by encoding
refinement parameters as uninterpreted propositions within the refinement logic.
We illustrate how this mechanism yields a variety of sophisticated means for
reasoning about programs, including: 
inductive refinements for reasoning about higher-order traversal routines,
compositional refinements for reasoning about function composition,
index-dependent refinements for reasoning about key-value maps, and
recursive refinements for reasoning about recursive data types.

\subsection{The key idea}
Consider the monomorphic @max@ function on @Int@ values.
We give @max@ a refinement type, stating that its result
is greater or equal than both its arguments:
$$\centering\begin{tabular}{c}
\begin{code}
max     :: x:Int -> y:Int -> {v:Int | v >= x && v >= y}
max x y = if x > y then x else y
\end{code}
\end{tabular}$$
If we apply @max@ to two positive integers, 
say @n@ and @m@, we get that the result is greater or equal to both 
of them, as @max n m :: {v:Int | v >= n && v >= m}@.
However, we can not reason about an arbitrary property: 
If we apply @max@ to two even numbers, can not verify that the result is 
also even.
%
Thus, even though we have the information that both arguments are even on the input, 
we lose it on the result.

To solve this problem, we introduce \emph{abstract refinements} 
which let us 
quantify or parameterize a type over its constituent refinements.
Using abstract refinements, we can type @max@ as
$$\centering
\begin{tabular}{c}
\begin{code}
max :: forall <p::Int->Bool>. Int<p> -> Int<p> -> Int<p>
\end{code}
\end{tabular}$$
where @Int<p>@ is an abbreviation for the refinement type {@{v:Int | p(v)}@}.
Intuitively, an abstract refinement @p@ is encoded in the refinement logic 
as an \emph{uninterpreted function symbol}, which satisfies the
\emph{congruence} axiom~\cite{Nelson81}
%
$$\forall \overline{X}, \overline{Y}: (\overline{X} = \overline{Y})
\Rightarrow P(\overline{X}) = P(\overline{Y})$$
%

It is trivial to verify, with an SMT solver, that @max@ 
enjoys the above type: the input types ensure that both @p(x)@ and @p(y)@ 
hold and hence the returned value in either branch satisfies 
the refinement  @{v:Int | p(v)}@, thereby ensuring the output 
type. 

In a call site, 
we simply instantiate
the \emph{refinement} parameter of @max@ with the concrete 
refinement,
after which type checking proceeds as usual. 
%
As an example, suppose that we call @max@ with two even numbers:
$$\centering
\begin{tabular}{c}
\begin{code}
n :: {v:Int | even v}
m :: {v:Int | even v}
\end{code}
\end{tabular}$$

Then, the abstract refinement can be instantiated with a concrete predicate 
@even@, 
which will give @max@ the type
$$\centering
\begin{tabular}{c}
\begin{code}
max [even] :: 
{v:Int | even v} -> {v:Int | even v} -> {v:Int | even v}
\end{code}
\end{tabular}$$
where the expression in brackets is the refinement instantiation.
% 
Since both @n@ and @m@ are even numbers, they satisfy function's preconditions, 
thus we can apply them to @max@, to get an even result:
$$\centering
\begin{tabular}{c}
\begin{code}
max [even] n m :: {v:Int |  even v}
\end{code}
\end{tabular}$$

This is the basic concept of abstract refinements, which, 
as we shall see have many interesting applications.

\subsection{Inductive Refinements}
As a first application we present, how abstract refinements 
allow us to formalize induction within the type system.

Consider a @loop@ function that takes as arguments
a function @f@, an integer @n@, 
a base case @z@ and applies 
the function @f@ to the @z@, @n@ times:
%
$$\centering
\begin{tabular}{c}
\begin{code}
loop :: (Int -> a -> a) -> Int -> a -> a
loop f n z = go 0 z
  where go i acc | i < n     = go (i+1) (f i acc)
                 | otherwise = acc 
\end{code}
\end{tabular}$$
%
Now, consider a user function @incr@ that uses @loop@
and at each iteration increases the accumulator by one:
$$\centering
\begin{tabular}{c}
\begin{code}
incr :: Int -> Int -> Int
incr n z = loop f n z
  where f i acc = acc + 1
\end{code}
\end{tabular}$$
%
The accumulator is initialized with @z@ and at each @loop@'s iteration
it is increased by @1@. So, at the @i@th iteration, the accumulator is equal to 
@z+i@. There will be @n@ iterations, thus the final result will be @z+n@.
%
This reasoning constitutes an inductive proof that characterizes @loop@'s behaviour.
However, 
it is unclear how to give @loop@ a (first-order) refinement type that 
describes its inductive behaviour.
Hence, it has not been possible to verify that @incr@ actually adds its two arguments.

\mypara{Typing loop}
Abstract refinements allow us to solve this problem, 
while remaining within the boundaries of SMT-based decidability.
We give @loop@ the following type:
$$\centering
\begin{tabular}{c}
\begin{code}
loop :: forall <r :: Int -> a -> Bool> .
           f : (i:Int -> a<r i> -> a<r (i+1)>) 
        -> n : {v:Int | n >= 0} 
        -> z : a<r 0> 
        -> a<r n>
\end{code}
\end{tabular}$$
The trick is to qualify over the invariant @r@ that @loop@
establishes between the loop iteration and the accumulator.
Then the type signature encodes induction on natural numbers:
(1) @n@ should be a natural number, thus a non-negative integer,
(2) the base case @z@ should satisfy the invariant at 0,
(3) in the inductive step, @f@ 
uses the old accumulator to create the new one, thus
if the old accumulator satisfies the invariant on the iteration @i@,
the new one, as constructed by @f@, should satisfy the invariant at @i+1@.
If these four conditions hold, we conclude that the result satisfies the invariant at @n@.
This scheme is not novel\cite{coq-book}; what is new is the encoding, via uninterpreted 
predicate symbols in a SMT-decidable refinement type system.

\mypara{Using loop}
We can use this expressive type of @loop@ to verify inductive properties of
user functions:
$$
\begin{tabular}{c}
\begin{code}
incr :: n:{v:Int|v >= 0} -> z:Int -> {v:Int|v = n + z}
incr n z = loop [{\i acc -> acc + i}] f n z
  where f i acc = acc + 1
\end{code}
\end{tabular}$$
%
In the above example, the expression in brackets denotes 
the instantiation of the abstract refinement.
For purpose of illustration 
we make abstract refinement instantiation explicit, 
but it could be automatically inferred via liquid typing\cite{Vazou13}.
%Moreover, 
%in \cite{Vazou13} we present how this inductive reasoning 
%can be applied to reason about structural inductive properties in lists.

\subsection{Function Composition}

As a next example, we present how one can use abstract refinements
to reason about function composition.

Consider a @plusminus@ function that composes a plus and a minus operator:
$$\centering
\begin{tabular}{c}
\begin{code}
plusminus :: n:Int 
          -> m:Int 
          -> x:Int 
          -> {v:Int | v = (x - m) + n}
plusminus n m x = (x - m) + n
\end{code}
\end{tabular}$$
In a first order refinement system we can verify that the function's behaviour
is captured by its type.
%
However, consider an alternative definition that uses function composition 
@(.) :: (b -> c) -> (a -> b) -> a -> c@.
$$\centering
\begin{tabular}{c}
\begin{code}
plusminus n m x = plus . minus
  where plus  x = x + n
        minus x = x - m
\end{code}
\end{tabular}$$
It is unclear how to give @(.)@ a (first-order) refinement type that
expresses that the result can be refined with the composition of the
refinements of both arguments results.
%
Thus, this definition of @plusminus@ can not have the previous descriptive type.

\mypara{Typing function composition}
To solve this problem, we can use abstract refinements and give @(.)@ a type:
$$\centering
\begin{tabular}{c}
\begin{code}
(.) :: forall < p :: b -> c -> Bool
              , q :: a -> b -> Bool>.
       f : (x:b -> c<p x>)
    -> g : (x:a -> b<q x>)
    -> x : a
    -> exists[z:b<q x>]. c<p z>
\end{code}
\end{tabular}$$
The trick is once again to quantify the type over 
refinements we care about.
This time, we use two abstract refinements:
the refinement @p@ of the result of the first function @f@
and the refinement @q@ of the result of the second function @g@.
For any argument @x@, we use an existential to bind the intermediate result to @z = g x@, 
so @z@ satisfies @q@ at @x@, 
and the result satisfies @p@ at the intermediate result.

\mypara{Using function composition}
With this type for function composition,
user functions get the concrete refinement of the final result
to be the composition of the two refinements of the argument functions.

Back to @plusminus@ example, with the appropriate refinement instantiation we get 
the concrete refinement type for function composition:
$$\centering
\begin{tabular}{c}
\begin{code}
(.) [{\x v -> v = x + n}, {\x v -> v = x - m}] 
    :: f : (x:b -> {v:c | v = x+n})
    -> g : (x:a -> {v:b | v = x-m})
    -> x : a
    -> exists[z:{v:b | v = x-m}]. {v:c | v = z+n}
\end{code}
\end{tabular}$$
The result type asserts that there exists a value @z@,
which is indeed the intermediate result, with the property @z = x - m@.
With this, the final result is equal to @z + n@.
If our logic supports equality, as SMT solvers do, 
we can verify that the final result is indeed equal to @(x - m) + n@.
In other words, we can verify the desired type of @plusminus@. 

\subsection{Index-Dependent Invariants}\label{sec:overview:index}

Next, we illustrate how abstract invariants allow us to 
specify and verify index-dependent invariants of key-value maps. 
To this end, we encode \emph{extensible vectors} 
as functions from @Int@ to 
some generic range @a@. Formally, we specify vectors as 
%
$$\centering\begin{tabular}{c}\begin{code}
data Vec a <dom :: Int -> Bool, rng :: Int -> a -> Bool> 
  = V (i:Int<dom> -> a <rng i>)
\end{code}\end{tabular}$$
%
Here, we are parameterizing the definition of the type @Vec@ 
with \emph{two} abstract refinements, @dom@ and @rng@, which
respectively describe the \emph{domain} and \emph{range} of the vector.
That is, @dom@ describes the set of \emph{valid} indices, 
and @rng@ specifies an invariant relating each @Int@ index
with the value stored at that index.

\mypara{Describing Vectors}
With this encoding, we can descibe various vectors. 
To start with we can have vectors of @Int@ defined on positive integers
with values equal to their index:
%
$$\centering\begin{tabular}{c}\begin{code}
Vec <{\v -> v > 0}, {\_ v -> v = x}> Int
\end{code}\end{tabular}$$
%
Or a vector that is defined only on index 1 with value 12:
%
$$\centering\begin{tabular}{c}\begin{code}
Vec <{\v -> v = 1}, {\_ v -> v = 12}> Int
\end{code}\end{tabular}$$
%
As a more interesting example, we can define a \textit{Null Terminating String}
with length @n@, 
as a vector of @Char@ defined on a range @[0, n)@ 
with its last element equal to the terminating character:
%
$$\centering\begin{tabular}{c}\begin{code}
Vec <{\v -> 0 <= v < n}
    ,{\i v -> i = n-1 => v = `\0`}> Char
\end{code}\end{tabular}$$
%
Finally, we can encode a \textit{Fibonacci memoization vector}, 
which can be used to efficiently compute a Fibonacci number,
that is defined
on positive integers and its value on index @i@ is either zero
or the @i@th Fibonacci:
%
$$\centering\begin{tabular}{c}\begin{code}
Vec <{\v -> 0 <= v}
    ,{\i v -> v != 0 => v = fib(i)}> Char
\end{code}\end{tabular}$$

\mypara{Using Vectors}
A first step towards using vectors is 
to supply the appropriate types 
for vector operations, like 
set, get and empty.
This usually means qualifying
over the domain and the range of the vectors. 
%
Then, the programmer has to specify interesting vector properties, 
as we did for the Fibonacci memoization, or the null terminating string.
% 
Finally, the system can verify that user functions, that transforms vectors
preserve these properties.
%
This procedure is applied in \cite{Vazou13},
where, with the appropriate types for vector operations,
we reason about functions that transform Null Terminating Strings
or efficiently compute a Fibonacci number.
%


\subsection{Recursive Invariants}
Finally, we turn our attention to recursively defined datatypes, and show 
how abstract refinements allow us to specify and verify high-level
invariants that relate the elements of a recursive structure.
Consider the following refined definition for lists:
%
$$\centering\begin{tabular}{c}\begin{code}
data [a] <p :: a -> a -> Bool> where
  []  :: [a]<p>
  (:) :: h:a -> [a<p h>]<p> -> [a]<p>
\end{code}\end{tabular}$$

The definition states that a value of type @[a]<p>@ 
is either empty (@[]@) or constructed from a pair of  
a \emph{head} @h::a@ and a \emph{tail} of a list of 
@a@ values \emph{each} of which satisfies the refinement @(p h)@. 
Furthermore, the abstract refinement @p@ holds recursively
within the tail, ensuring that the relationship @p@ 
holds between \emph{all} pairs of list elements.

Thus, by plugging in appropriate concrete refinements, 
we can define the following aliases, which correspond 
to the informal notions implied by their names:
$$\centering\begin{tabular}{c}\begin{code}
type IncrList a = [a]<{\h v -> h <= v}>
type DecrList a = [a]<{\h v -> h >= v}>
type UniqList a = [a]<{\h v -> h != v}>
\end{code}\end{tabular}$$
%
That is, @IncrList a@ (resp. @DecrList a@) describes a list sorted
in increasing (resp. decreasing) order, and @UniqList a@ describes
a list of \emph{distinct} elements, \ie not containing any duplicates.
We can use the above definitions to verify
%
$$\centering\begin{tabular}{c}\begin{code}
[1, 2, 3, 4] :: IncrList Int
[4, 3, 2, 1] :: DecrList Int
[4, 1, 3, 2] :: UniqList Int
\end{code}\end{tabular}$$
%
More interestingly, we can verify that the usual algorithms 
produce sorted lists:
%
$$\centering\begin{tabular}{c}\begin{code}
insertSort :: (Ord a) => [a] -> IncrList a 
insertSort []     = []
insertSort (x:xs) = insert x (insertSort xs) 

insert :: (Ord a) => a -> IncrList a -> IncrList a 
insert y []       = [y]
insert y (x:xs) 
  | y <= x        = y : x : xs
  | otherwise     = x : insert y xs
\end{code}\end{tabular}$$
%
Thus, abstract refinements allow us to \emph{decouple} the definition 
of the list from the actual invariants that hold.
This, in turn, allows us to conveniently reuse the same 
underlying (non-refined) type to implement various algorithms 
unlike, say, singleton-type based implementations which require 
up to three different types of lists (with three different ``nil" and ``cons" 
constructors~\cite{Sheard06}). This, makes abstract refinements 
convenient for verifying complex sorting implementations like that of 
@Data.List.sort@ which, for efficiency, use lists with different 
properties (\eg increasing and decreasing).

\subsection{Implementation}
We implemented abstract refinements in HSolve, 
a tool that uses liquid types as a base refinement system.
HSolve takes an input a haskell source code and some specifications.
If it can prove that the code satisfies the specifications, 
it returns \textit{Safe} combined with the code annotated with the inferred
refinement types.
Otherwise, it returns \textit{UnSafe}, combined with the source location 
that could not be verified.

In the tool abstract refinements appear only on type specifications
and are transparent for the haskell code.
Refinement abstraction inserts a variable which is treated as an uninterprented functions
symbol by the SMT solver. 
In refinement application the concrete refinement is inferred via liquid 
type variables; thus the user should not explicitly annotate the code with refinement instantiations.


We use the tool to verify 
some benchmarks, among which ghc official sorting algorithm on lists
and two ghc libraries (Data.Map.Base and Data.Set.Splay) that reason about Binary Search Trees.
The specification invariants are usually simple, as abstract refinements
greatly simplify writing specifications for the majority of interface or public functions.
Moreover, the verification procedure required only a few code modifications, 
which include reordering arguments or inserting ghost variables, 
to make explicit values over which various invariants depend.

\subsection{Formal Language}
We suggest that any refinement system can be extended with abstract refinements
without increasing its complexity.
%
First of all, the syntax should be extended to support refinement abstraction
and application:		
If refinement abstraction, we abstract from an expression $e$
the refinement $\pi$ with type $\tau$, while in refinement application
we instantiate an abstrast refinement with a concrete one $p$
that may have some parameters $\bar{x}$.
%
Then, the predicates of the language should be extended to 
include abstract refinements, applies to program expressions
%
The types of the language should also be extended to include 
refinement abstraction.
%

Since we extended our expressions the relevant typing rules should be added:
The refinement abstraction expression is typed as an refinement abstraction
type: the abstract refinement is treated as a variable
and the checking proceeds in a straightforward way.
In the refinement application, the abstract refinement $\pi$ is replaced with a concrete one
over the type $\tau$. A formal definition of this substitution can be found in our paper.

Similarly, since we extended our types, the wellformdness and subtyping 
rules should be extended.
In both cases, the abstract refinement is added in the environment
and the check proceeds in a straightforward way.

We note that
abstract refinements 
can be treated as uninterprented functions in the implication
checking algorithm, thus the complexity of the system is not increased.
Moreover, they appear only in the types, thus they can be erased in run-type.
\begin{figure}[ht!]
\centering
$$
\begin{array}{rrcl}
\emphbf{Expressions} \quad 
  & e 
  & ::= 
  & 		 \dots
  \spmid \epabs{\rvar}{\tau}{e}
  \spmid \epapp{e}{\efunbar{x:\tau_x}{p}} 
  \\[0.05in] 

\emphbf{Predicates} \quad 
  & p
  & ::= 
  &		\dots
  \spmid \rvapp{\rvar}{e}  
  \\[0.05in] 
\emphbf{Types} \quad 
  & \tau 
  & ::= 
  &		 \dots
  \spmid \tpabs{\rvar}{\tau}{\tau}

\end{array}
$$
\caption{\textbf{Syntax of Expressions, Types and Schemas}}
\label{fig:syntax}
\end{figure}


\begin{figure}[ht!]

\medskip \judgementHead{Type Checking}{$\hastype{\Gamma}{e}{\tau}$}
$$\begin{array}{cc}
\inference
    {\hastype{\Gamma, \rvar:\tau_\rvar}{e}{\tau} &&
     \isWellFormed{\Gamma}{\tau} 
    }
    {\hastype{\Gamma}{\epabs{\rvar}{\tau_\rvar}{e}}{\tpabs{\rvar}{\tau_\rvar}{\tau}}}
    [\tpgen]
&  

\inference
    {\hastype{\Gamma}{e}{\tpabs{\rvar}{\tau_\rvar}{\tau}} && 
     \hastype{\Gamma}{\efunbar{x:\tau_x}{p}}{\tau_\rvar}
    }
    {\hastype{\Gamma}
             {\epapp{e}{\efunbar{x:\tau_x}{p}}}
             {\rpinst{\tau}{\rvar}{\efunbar{x:\tau_x}{p}}}
    }
    [\tpinst]
\end{array}$$

\medskip \judgementHead{Subtyping}{\isSubType{\Gamma}{\tau}{\tau}}
$$
\inference
   {(\Gamma, v:b \vdash \texttt{Valid}(p_1 \Rightarrow p_2))}
   {\isSubType{\Gamma}{\tref{b}{p_1}}{\tref{b}{p_2}}}
   [\tsubBase]
$$
\caption{\textbf{Static Semantics} from $\lambda_C$ to $\lambda_A$}
\label{fig:rules}
\end{figure}
